% !TeX root = ../main.tex

\chapter{绪论}

\section{研究背景和意义}
区块链作为一种去中心化的分布式账本,随着技术的不断发展,受到越来越多的关注,正应用于许多领域,如金融部门,加密货币交易所,游戏,医疗保健和健康应用场景等。区块链技术最早是被引用于比特币中,它作为底层技术,提供了去中心化、不可篡改、透明性、匿名性等特性,让比特币能够在去中心化的环境中实现交易。由于区块链提供的这些特性,让其能够解决许多信任与安全的问题,引起了学术界和工业界广泛研究和应用。

同时随着区块链应用场景逐渐增多以及复杂,智能合约被引用到区块链中,如以太坊。智能合约作为一种基于区块链技术的自动化合约,其中包含了可以自动执行、操纵或验证合同条款的计算机程序。这些合约通过代码的形式被部署在区块链上,而不是传统合同中的法律文本。智能合约的执行是由区块链网络上的节点来完成的,保证了透明性、不可篡改性和去中心化的特性。因此,智能合约的引入为区块链技术带来了非常广阔的应用场景,也受到了各国的重视,比如我国的“十四五”规划纲要中强调了加强区块链技术创新,特别是智能合约、共识算法、加密算法和分布式系统等方面的创新。着重发展联盟链,并将区块链服务平台应用于金融科技、供应链管理、政务服务等领域。此外,规划还提出了完善监管机制,以确保区块链产业的健康发展。

随着区块链技术的广泛应用,当前区块链系统的可扩展性差的性能瓶颈问题逐渐暴露了出来,大大限制了区块链技术的进一步应用和发展。像我们熟知的比特币,每秒钟最多只能处理7笔交易。以太坊的吞吐量则大约为17笔交易每秒。区块链系统的低吞吐量显然无法满足日益增多的交易需求。相对于主流金融系统如微信支付、支付宝和Visa等,区块链的吞吐量较低,难以满足互联网规模的用户支付需求,更不用说在物联网场景下,对于高频数据交换所需的高吞吐量。因此,提升区块链的可扩展性对于加速实际应用的落地至关重要,以满足不断增多的用户和数据交换的需求。

\section{国内外研究现状}
区块链作为分布式账本,允许在互不信任的参与者之间实现共识和维护一致的状态。为了实现这一目标,区块链依赖于分布式共识协议,这些协议包括工作证明(Proof-of-Work, PoW),比如比特币和拜占庭容错(Byzantine Fault Tolerance, BFT),比如长安链等。这些共识机制确保了区块链网络的稳定性和安全性,使得参与者能够在无需相互信任的情况下共同维护整个系统的一致性状态。

共识协议一般可分为两个阶段:首先,一个区块提出者创建一个包含交易列表的新区块,并将其广播到网络;接着,其他节点验证接收到的区块,如果区块有效,就执行并将其记录到本地账本。通过共识协议,能够保证块提出者的状态和其他节点的状态是一致的,实现分布式账本。为了实现严格的一致性,传统的区块链系统以串行方式执行区块的交易,即按照交易在区块中的顺序,串行执行。然而,这种串行执行方案严重限制了处理事务的能力,导致系统性能较差。

因此,为了提高区块链的可扩展性,使得区块链的吞吐量不再是性能瓶颈,各界研究人员正积极探索并行执行区块链中的交易的解决方案。总的来说目前的解决方案可以分为两大类,一类是链上解决方案,另一类则是链下解决方案。

\subsection{链上解决方案}
传统的串行执行区块链交易并未充分利用现代计算机的多核CPU的性能,只利用了单核CPU,降低了交易的执行速度。因此为了提升区块链中交易的执行速度,研究人员利用悲观或者乐观锁,以及交易调度关系图,让区块链中的交易可以并行执行,以此来充分利用CPU的多核能力。因此链上的解决方案的主要思想是在共识的第一阶段引入一种并发调度算法,通过该算法,块提出者可以生成一个交易的调度关系,用于验证器并发执行事务。

\subsubsection{悲观锁解决方案}
为了让交易能够并行执行,并且没有数据冲突,Dickerson等人将抽象锁机制引入到了区块链的交易执行中,并且每个存储操作都有一个抽象锁,防止数据冲突。区块提出者将推测性地并行执行交易,并允许不冲突的合约并行执行。结合反向日志机制在执行过程中能够发现交易的可串行化的并发调度。调度信息被描述为一个有向无环图。验证节点结合调度信息以及fork-join程序来确定性的执行区块中的交易,保证节点之间状态的一致性。

\subsubsection{乐观锁解决方案}
除了利用悲观锁机制防止数据冲突外,[7][6]等人利用乐观锁机制并发执行区块链的交易,即让交易先执行,当交易发现冲突后,回退有冲突的交易,防止交易的数据冲突。为了让主节点和其他验证节点之间状态一致,[7]设计了交易拓扑关系图(TDG, Transaction Dependency Graph)。[6]设计了具有确定性中止的乐观并发控制(OCC-DA, Optimistic Concurrency Control with Deterministic Aborts),让乐观并发控制在公链中的应用成为了可能。

然而无论是使用悲观锁还是乐观锁,区块链中的每个节点仍然需要执行所有交易。当一个交易是计算密集型的,即对CPU的资源需求比较高(例如,智能合约交易[19]),吞吐量的提高是有限的。因为计算时间最长的交易会限制最终节点提交的时间,如图所示。当Tx1的交易时间远远超过其他交易的时间的时候,并行执行交易对系统带来的吞吐量是有限的。

除此之外,热点账户的存在(即频繁访问的账户)可能导致高数据争用,如以太坊之前的加密猫[],因此节点仍然需要串行执行事务。如图所示。当所有交易都是关于账户a的操作的时候,交易没有办法并行执行,只能串行执行,此时并行执行交易没有办法提升系统的吞吐量。

\subsection{链下解决方案}
为了更好地解决计算密集型交易和热点账户的问题,让交易能够更好的并发执行,其他工作[],提出了链下并发执行方案。链下解决方案的核心思想是将一些交易放到链下节点中执行,通过这样的方法让链上节点和链下节点能够并行执行不冲突的交易。链上节点可以直接使用链下节点的执行结果来达成共识。为了让链上节点信任链下节点的执行结果,需要对链下节点执行智能合约的方法进行一定的设计,而不是和链上节点一样直接执行智能合约然后将结果返回给链上节点。一般执行方案可以分为两种,乐观链下扩容和悲观链下扩容。

\subsubsection{悲观链下扩容}
悲观链下扩容的核心思想是将交易放到链下执行,利用密码学或者硬件的方式,对链下执行的交易生成一定的执行证明。链上节点通过验证执行证明,来确认并保证链下节点是按照正确的执行过程执行交易,而没有恶意篡改交易执行结果。

常用的密码学实现悲观链下扩容的方式有零知识证明(ZKPs,Zero-Knowledge Proofs)。它允许一方(证明者)向另一方(验证者)证明其拥有某个信息或某个声明是真实的,而无需透露实际的信息细节。ZK Rollup是一种基于零知识证明的扩容方案。在ZK Rollup中,链下交易被分批处理,并生成加密的有效性证明,以验证每批交易的真实性和有效性。这些有效性证明被添加到链上节点中,从而将批量交易合并到链上。一旦提交,这些证明可以由链上的智能合约迅速验证,从而有效地处理和确认交易批次。此外,ZK扩容方案采用多种证明策略,例如SNARK、STARK、PLONK和DARK。它们在数据足迹、证明时间、验证时间、协同风险等方面具有各自的安全性和风险特征。

除了使用密码学中的方法,还有依靠硬件可信执行环境(TEE, Trusted Execution Environments),如intel的SGX(Software Guard Extensions)和ARM的TrustZone,来让帮助链上节点验证链下节点的执行结果。Ekiden[8]为智能合约提供了保密性,并通过因特尔的SGX的enclave执行链下合约,将合约执行与共识流程脱钩。FASTKITTEN利用可靠的执行环境(TEE),为任意复杂的智能合约提供了全面支持。FASTKITTEN通过使用TEE来隔离enclave内的合同执行,保护其免受潜在恶意用户的攻击。

\subsubsection{乐观链下扩容}
乐观链下扩容借助密码学或者硬件,基于乐观机制,使用欺诈证明。在该方案中,系统乐观地相信交易数据的正确性,而无需对每笔交易进行验证。交易提交后会进入一个等待期,在等待期内,如果没有节点提出异议并提供证据证明存在恶意交易,那么交易将被自动确认并写入区块。乐观链下扩容比较突出的方案是Arbitrum和Optimism,它们结构上相似,但在多方欺诈逻辑和处理兼容性问题上略有差异。

Optimism采用的是单轮非交互式欺诈证明。验证者将不带可验证证明的状态承诺发送到以太坊上,并质押一定的保证金。这些承诺会有一个挑战期,在挑战期内视为待处理。如果有人对区块的交易数据提出异议,则需发起挑战且同样数量的质押保证金,即可以通过恶意证明流程使它无效。然后,协议将在链上上重新计算所有交易以判定对错,错误方将被罚没保证金,而正确方将获得奖励。当挑战成功的时候它会从状态确认链中删除,然后被另一个状态承诺替代。即使挑战成功,Optimism也不会回滚,Optimism的状态和交易顺序不会因为恶意证明挑战成功而修改,只会发布链的状态承诺。如果这个承诺在挑战期内没有被挑战,那么这个承诺就会被视为已确认,以太坊上的智能合约就可以安全的接收基于该承诺的状态证明。

相比于Optimism采用单轮非交互欺诈证明,Arbitrum采用多轮交互型欺诈证明。验证者同样将压缩后的数据同步到链上节点,并质押一定的保证金。当存在节点对区块的交易数据提出异议时,验证者和挑战者在链下通过二分法逐步拆分争议步骤,最终在链上节点对该步骤进行判定,从而高效解决争议。因为Arbitrum需要使用二分法不断确认存在争议的步骤,具体而言对应的是Arbitrum虚拟机中的一条指令。并且在二分的过程中会增加交互次数,会增加系统的整体复杂度。除此之外,交互式欺诈证明需要多方参与,让最终挑战完成的时间大于Optimism挑战完成的时间。而Arbitrum的优点则在于只用验证单步,链上验证的工作量比较小,能够以较低的链上成本解决争议。

除了欺诈证明的链下扩容之外,Liu等人基于传统的拜占庭容错(BFT, Byzantine Fault Tolerance)[16]设计了Saber范式,将执行与共识分开。不同组的执行节点可以并行执行交易;同时,共识节点可以异步排序交易和处理执行结果。而且,它不需要执行节点之间的协调,可以有效地防止活锁。Das等人提出了一种支持异步执行合约调用的随机采样模型YODA[17]。对于每个合约调用,随机选择一个矿工子集来独立于挖矿过程执行合约代码,并以新交易的形式返回结果。矿工子集的大小需要通过基于似然估计的多轮自适应一致性算法(MIRACLE, MultI-Round Adaptive Consensus using Likelihood Estimation)来确定。在MIRACLE算法中,矿工计算每个接收到的摘要的可能性,这主要取决于不同摘要的数目和拜占庭节点的数目。如果任何摘要的可能性超过特定阈值,MIRACLE将宣布其相应的解决方案为正确的。否则,它迭代地选择额外的计算集合,直到某个摘要的可能性越过所需的阈值。

综上所述,能够发现,目前智能合约执行方案可以分成两类,一类是完全把智能合约放到链上执行,并根据区块链中智能合约需要两阶段执行的特点将数据库中的并行执行方案用到智能合约的执行中。这些相关工作分别考虑两个阶段,即它们简单地采用成熟的并发控制协议来发现冲突可序列化调度,而忽略调度信息的一些关键属性,例如有向无环图的并行性。例如,在Dickerson[5]的研究中,主节点可能会发现一个可序列化的顺序,其中每个交易都与其先前的交易发生冲突,即生成一个链图。除此之外,这些相关的工作并没有考虑系统中节点的性能。区块链系统中节点的性能不一定是均衡的,可能存在性能比较差的节点。即使引用针对区块链系统的并行技术,而若系统中节点的性能比较差,仍然无法提高系统的整体性能。这是因为区块链系统需要达成共识,而若要完成共识,则需要系统中参与共识的验证节点根据主节点的发布的区块重新执行一遍交易,来完成对主节点状态的验证。而若系统中存在性能较差的节点,当区块打包的交易数目多时,会导致性能较差的节点无法在设计的共识时间内完成投票,使得节点无法参与共识,并让区块验证无法通过,最终影响系统的性能。

另一类则是将智能合约的执行放到链下,减少链上节点的计算负担,提高系统的性能。因为链下节点不参与链上的共识,并且链上的节点不再执行智能合约,此时要验证智能合约执行的结果是否正确,即链下节点是否诚实地执行智能合约,是链下计算的关键挑战点。然而目前设计的链下计算存在一定的问题。YODA系统的主要限制是每个智能合约调用都需要由单独的子集执行多轮,并且采样的子集需要相对较大(例如,数百个节点)以减少作弊的可能性。与标准以太坊相比,这些方法可以降低所需的执行的重复程度(例如,从数千个节点到数百个节点),但它们仍然需要大量的冗余,多轮执行过程,这会导致巨大的通信开销,并且完整性破坏的概率不可忽略。此外,像YODA这样的系统需要一个无偏分布式随机信标,如 RandHound[19],负担较大。在Arbitrum中如果所有指定的链下执行者都签署了相同的结果,则执行结果将被矿工接受。如果不是所有交易执行者都签署结果,则交易不会立即被接受,而是进入挑战期。签名的执行者和有争议的挑战者需要支付保证金,然后其中之一使用高效的二分法挑战机制证明其结果的正确性。在挑战期间内,合约不能取得进展。最严重的情况可能是在挑战期内,如一个月,合约都无法被调用。同时,在Arbitrum中跨合约调用不是原子执行的,不能保证可串行性,长调用链会产生高延迟。Optimism虽然采用的是单轮非交互式欺诈证明,交互过程相比于Arbitrum简单了很多,但如果有挑战,则需要把区块中的所有交易都重放执行一遍,重放成本会比较高。当需要重新计算你的交易过多的时候,甚至可能会受到链上区块大小的限制。Ekiden可以执行更复杂的合约,但 Ekiden要求所有enclave都是可信的,因为破坏一个 enclave 会使对手可以任意地违反合约的完整性。除此之外,最近的研究表明,TEE本身也存在一定的安全隐患[20–24]。对于Saber而言,它改变了智能合约的编码范式:合约开发者在开发合约时需要枚举所有的依赖关系,这大大增加了智能合约的编码难度。

因此,目前智能合约的执行方案存在一定的问题,无法兼顾性能与安全,亟须提供一种能够让智能合约的安全和执行的性能得到更好的保障的方法。
\section{研究内容与创新点}


\section{论文组织架构}


